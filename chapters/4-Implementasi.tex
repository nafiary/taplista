\chapter{IMPLEMENTASI}
\label{chapter:implementasi}

Pada bab ini dijelaskan mengenai implementasi \textit{publish-subscribe} pada rancang bangun sistem \textit{monitoring} perangkat di DPTSI ITS.

\section{Lingkungan Implementasi}
Lingkungan implementasi dalam pembuatan Tugas Akhir ini meliputi perangkat keras dan perangkat lunak yang digunakan untuk rancang bangun sistem \textit{monitoring} perangkat di DPTSI ITS dengan implementasi \textit{publish-subscribe} adalah sebagai berikut:

\begin{enumerate}
	\item Perangkat Keras:
	\begin{itemize}
		\item \textit{Processor} Intel(R) Core(TM) i5-4210U CPU @1.70GHz 64-bit dengan memori RAM 4 GB.
	\end{itemize}
	\item Perangkat Lunak:
	\begin{itemize}
		\item Sistem operasi Ubuntu 14.04 LTS 64 bit.
		\item Rabbitmq, digunakan sebagai broker pada \textit{publish-subscribe}.
		\item Flask, digunakan untuk membuat aplikasi sistem.
		\item Nagios, digunakan untuk memantau server.
		\item MySQL, digunakan untuk membangun \textit{database} pada sistem.
		\item Telegram, digunakan sebagai media penerimaan notifikasi dari sistem.
		\item \textit{Text editor} Sublime Text 3.
		\item TeXstudio, digunakan untuk menyusun buku tugas akhir ini.
		
	\end{itemize}
\end{numerate}
	
\section{Implementasi \textit{Publisher Server}}
	\textit{Publisher server} merupakan \textit{server} yang berfungsi untuk mengambil data pada perangkat (\textit{server}) dan mengirimkannya menuju \textit{publish-subscribe server}. \textit{Publisher server} menggunakan \textit{plugin} NRPE sebagai tambahan pada Nagios untuk memantau sumber daya "lokal" (seperti beban CPU, penggunaan memori, dan sebagainya). Untuk itu kita perlu memasang Nagios terlebih dahulu pada \textit{server} agar bisa melakukan pengambilan data pemantauan \textit{server}. 
	
	Setelah data berhasil dikumpulkan, data hasil pemantauan pada tiap \textit{server} dikirimkan menuju \textit{publish-subscribe server} melalui \textit{thread} yang berbeda. Proses ini dinamakan \textit{multithreading}.
	
	\subsection{Instalasi Nagios pada \textit{Server}}
	Instalasi dilakukan dengan mengunduh Nagios terlebih dahulu pada laman di bawah ini
	\url{https://assets.nagios.com/downloads/nagioscore/releases/nagios-4.3.4.tar.gz}.
	Setelah itu, dilakukan prosedur penginstalan Nagios di \textit{server}.
	
	\subsection{\textit{Script} NRPE untuk Pengambilan Data}
	Kode Sumber \ref{nrpe} ini menunjukkan \textit{script} dari \textit{[plugin]} NRPE untuk megambil data dari hasil pemantauan oleh Nagios.
	
\begin{lstlisting}[frame=single,breaklines,caption={Perintah Mengumpulkan Data Perangkat dengan NRPE},label=nrpe, captionpos=b]
$ /usr/local/nagios/libexec/check_nrpe -H <alamat_server> -c <nama_service> ['nama_command']
\end{lstlisting} 
	
	
	\section{Implementasi \textit{Publish-Subscribe Server}}
	Pada \textit{publish-subscribe server}, dipasang aplikasi RabbitMQ. RabbitMQ menerima seluruh data yang dikirim oleh \textit{publisher}. Setelah itu, RabbitMQ
	menyimpan dan menunggu hingga ada konsumen yang
	meminta data tersebut pada RabbitMQ . Data yang dikirimkan harus sesuai dengan apa yang diminta oleh konsumen, tidak boleh berbeda dari kriteria yang diminta. 
	
	Instalasi RabbitMQ dapat dilihat melalui halaman web resminya, \url{https://www.rabbitmq.com/install-debian.html}. Langkah instalasi diawali dengan penginstalan erlang, sebuah bahasa pemrograman yang dibutuhkan untuk menjalankan RabbitMQ. Setelah itu barulah dipasang \texttt{rabbitmq-server}. 
	
	Setelah RabbitMQ \textit{server} terpasang, selanjutnya dilakukan konfigurasi untuk mengaktifkan akses ke RabbitMQ Management Console, web admin milik RabbitMQ. Hal ini dilakukan agar mudah untuk melakukan manajemen data, \textit{user}, dan lain-lain. RabbitMQ sudah menyediakan \textit{plugin} agar web admin dapat langsung digunakan. Hanya dengan menjalankan perintah \texttt{sudo rabbitmq-plugins enable rabbitmq\_management}, web admin RabbitMQ dapat dijalankan. 
	
	\section{Implementasi \textit{Webserver}}
	Pada \textit{webserver}, terdapat \textit{application server} dan \textit{websocket.} \textit{Websocket} berfungsi untuk meminta data dari \textit{publish-subscribe server}. Websocket disambungkan dengan suatu \textit{endpoint} pada \textit{application server}, sehingga ketika klien mengakses \textit{endpoint} pada aplikasi, \texttt{javascript} pada halaman tersebut akan menyambungkan ke halaman pada \textit{websocket}.
	
	\textit{Websocket} pada sistem ini menggunakan node.js, sebuah perangkat lunak yang didesain untuk mengembangkan aplikasi berbasis web. Node.js menggunakan bahasa pemrograman JavaScript.\\
	
	\begin{lstlisting}[frame=single,caption={Inisiasi Komunikasi Websocket dengan Klien dan Publish-Subscribe Server Tidak Terkonesi}, breaklines=true,label=pseudo:notconn, language=python]
	socket.on('disconnect', function() {
	$("#idlogs").text("Disconnect");
	$("#idlogs").parent().parent().attr('class', 'block-content block-content-full bg-danger');
	$('.nrperesult').each(function() {
	$(this).html("")
	});
	});
	\end{lstlisting}
	
	Kode Sumber \ref{pseudo:notconn} menunjukkan jika kondisi komunikasi antara \textit{websocket} dengan klien dan \textit{publish-subscribe server} tidak tersambung.
	
	\begin{lstlisting}[frame=single,caption={Inisiasi Komunikasi Websocket dengan Klien dan Publish-Subscribe Server Terkoneksi}, breaklines=true,label=pseudo:conn, language=python]
	socket.on('connect', function() {
	$("#idlogs").text("Connected");
	$("#idlogs").parent().parent().attr('class', 'block-content block-content-full bg-success');
	
	var deviceid = [];
	$('.device').each(function( index ) {
	deviceid.push($( this ).attr('id'));
	});
	socket.emit('startRabbit', {'data': deviceid, 'id': uuid4() });
	
	});
	\end{lstlisting}
	
	Kode Sumber \ref{pseudo:conn} menunjukkan jika kondisi komunikasi antara \textit{websocket} dengan klien dan \textit{publish-subscribe server} telah tersambung. Kemudian, klien mengirimkan ID dari perangkat (\textit{server}) yang telah dipilih oleh pengguna untuk menjadikannya nama \textit{exchange}. UUID versi 4 dibuat untuk menamai \textit{queue} yang baru. 
	
	Jika koneksi sudah tersambung dan ID perangkat untuk penamaan \textit{exchange} serta UUID versi 4 untuk penamaan \textit{queue} sudah dikirim oleh klien, maka selanjutnya \textit{server} akan memproses pembuatan \textit{exchange} dan \textit{queue} baru. Kode \textit{websocket server} saat pembuatan
	\textit{exchange} dan \textit{queue} lalu menyalurkan data pada klien dapat dilihat pada Kode Sumber \ref{pseudo:buatbaru}
	
\begin{lstlisting}[frame=single,caption={Pembuatan Exchange dan Queue Baru pada Websocket}, breaklines=true,label=pseudo:buatbaru, language=python]
rabbitMqConnection.createChannel().then(function(ch) {
consumerChannel = ch
var q;
function allDone(notAborted, arr) {
var okok = ch.prefetch(1)
okok = okok.then(() => {
return ch.consume(q, logMessage, {noAck: true}) 
.then(() => {
console.log(' [*] Waiting for '+msg['data']+'.');
})
})
}
forEach(msg['data'], function (item, index, arr) {
var done = this.async()
var ok = ch.assertExchange(item, 'fanout', {durable: false});
ok = ok.then(function() {
return ch.assertQueue(msg['id'], {exclusive: false});
});
ok = ok.then(function(qok) {
return ch.bindQueue(qok.queue, item, '').then(function() {
q = qok.queue
return qok.queue;
});
});
ok = ok.then(function() {
done()
})
}, allDone)
\end{lstlisting}
	
	\section{Implementasi REST API}
	Fungsi utama dari REST API untuk sistem ini adalah untuk menyimpan data pengguna yang melakukan \textit{subscribe} pada perangkat beserta \textit{service}-nya. REST API dibuat dengan menggunakan \textit{framework} Flask dan ORM Database peewee.
	
	REST API menggunakan protokol HTTP dalam pengaksesannya. Sehingga, terdapat URL API yang dapat disebut sebagai \textit{endpoint}. Tabel \ref{tab:endpoint} akan menguraikan ada \textit{endpoint} apa saja pada REST API di sistem ini. 
	
	\begin{longtable}{|p{0.05\textwidth}|p{0.40\textwidth}|p{0.13\textwidth}|p{0.25\textwidth}|} % L = Rata kiri untuk setiap kolom, | = garis batas vertikal.
		
		% Kepala tabel, berulang di setiap halaman
		\caption{Daftar Endpoint pada REST API} \label{tab:endpoint} \\
		\hline
		\textbf{No} & \textbf{\textit{Endpoint}} & \textbf{Metode} & \textbf{Aksi} \\ \hline
		\endhead
		\endfoot
		\endlastfoot
		1 & /register & POST & Membuat data baru pada tabel user di database \\ \hline
		2 & /login & POST & Mengambil data pada tabel user dan mencocokkannya dengan JSON yang dikirimkan lewat \textit{body}. setelah data username dan password cocok, lalu dibuatkan sebuah token JWT. \\ \hline
		3 & /logout & POST & Memasukkan token JWT yang terdaftar pada server kedalam daftar hitam agar token tidak dapat digunakan lagi. \\ \hline
		4 & /users & GET & Menampilkan seluruh data user yang terdaftar pada sistem \\ \hline
		5 & /users/\textless{}string:username\textgreater{} & GET & Menampilkan data user berdasarkan username yang tertulis pada URL \\ \hline
		6 & /devices/create & POST & Membuat data baru pada tabel devices di database \\ \hline
		7 & /devices/edit/\textless{}string:id\textgreater{} & PUT & Mengubah data pada tabel devices di database yang ID nya sama dengan ID yang ada pada URL. \\ \hline
		8 & /devices/delete & DELETE & Menghapus data pada tabel devices di database yang ID nya tertulis pada \textit{body} yang bertipe JSON. \\ \hline
		9 & /devices & GET & Menampilkan seluruh data perangkat yang terdaftar pada sistem \\ \hline
		10 & /devices/\textless{}string:id\textgreater{} & GET & Menampilkan data user berdasarkan username yang tertulis pada URL \\ \hline
		11 & /service/create & POST & Membuat data baru pada tabel service \\ \hline
		12 & /service/edit & POST & Mengubah data pada tabel service di database yang ID nya tertulis pada \textit{body} yang bertipe JSON. \\ \hline
		13 & /service/delete & POST & Menghapus data pada tabel service di database yang ID nya tertulis pada \textit{body} yang bertipe JSON. \\ \hline
		14 & /subscribe/devices & POST & Membuat data baru pada tabel subscribe \\ \hline
		15 & /unsubscribe/devices & POST & Menghapus data pada tabel subscribe di database yang ID nya tertulis pada \textit{body} yang bertipe JSON. \\ \hline
		16 & /subscribe/service & POST & Membuat data baru pada tabel servicesubscribed \\ \hline
		17 & /unsubscribe/service & POST & Menghapus data pada tabel servicesubscribed di database yang ID nya tertulis pada \textit{body} yang bertipe JSON. \\ \hline	
	\end{longtable}
	

